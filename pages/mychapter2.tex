% Chapter 2

\chapter{导航卫星精密轨道确定的基础原理}

\section{引言}

正如前述所说,提供实时高精度的导航卫星轨道服务对当前实时高精度定位具有重要意义。而导航卫星精密轨道确定的核心任务,就是利用长时间连续弧段的高精度GNSS观测值信息,确定导航卫星在该弧段内的空间位置信息。时至今日,导航卫星精密轨道确定理论也随着人们对其认识逐步的丰富和提高。

介绍几何法定轨和动力学法定轨。

接下来要介绍的主要内容简介

应该首先是GNSS精密定轨涉及的核心问题(不用背景),然后引入基本原理

\section{时空参考系统}

\subsection{时间系统}

\subsection{坐标系统}

\section{GNSS观测方程及其误差模型改正}

\subsection{函数模型}

\subsection{误差模型}

\subsection{随机模型}

\section{导航卫星运动模型}

\subsection{动力学方程和状态转移}

\subsection{摄动力模型}
