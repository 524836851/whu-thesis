% Chapter 2

\chapter{导航卫星精密轨道确定的基础原理}

\section{引言}

正如前述所说,提供实时高精度的导航卫星轨道服务对当前实时高精度定位具有重要意义,而导航卫星精密轨道确定的核心任务,就是利用长时间连续弧段的高精度GNSS观测信息,确定导航卫星在该时间弧段内的空间位置信息。
因此轨道确定中的其中一个常用的方法即为几何学定轨法(Bisnath et al. 1999;韩保民,2003) ,其核心原理与定位过程类似,利用高精度GNSS观测值直接交汇计算出目标体的动态位置。这类方法常常用于低轨卫星的轨道确定中,而对导航卫星,受限于地面观测几何构型,使用效果不足以满足当前对轨道精度的需求。

除利用高精度GNSS观测信息外,导航卫星自身的运动规律信息同样可以应用于定轨中。
考虑到导航卫星在空间环境中主要受到地球的万有引力作用,因而其具有围绕地球的类椭圆的周期性的运动轨迹。若将卫星和地球均视为质点,此时导航卫星的运动可被简化为一个二体问题。根据天体力学原理,其运动轨迹可被六个轨道参数所确定,此时轨道确定问题转化为轨道参数的确定。然而在导航卫星实际的运动过程中,除主要的地球万有引力,还会受到到其他摄动力的影响(如地球非球形引力、地球潮汐引起的引力摄动、地球辐射等等),此时将难以构建类似二体问题中轨道参数的解析表达式描述卫星的运动轨迹。在这种情况下,假定受力模型已知,导航卫星的运动轨迹信息可以通过数值积分的得到。动力学定轨法(刘林,1992)正是结合了导航卫星的运动模型和GNSS几何观测信息进行轨道确定,其通过动力学方程来维系导航卫星连续弧段内的位置信息,其显著改善了仅使用GNSS观测信息定位所带来的误差,因此导航卫星动力学模型的构建是动力学定轨法中的一个关键部分。

接下来本章就针对对导航卫星动力学定轨法中所涉及的基础算法原理进行相应梳理和介绍。

\section{时空参考系统}

bababab

\subsection{时间系统}

\subsection{坐标系统}

\section{GNSS观测方程及其误差模型改正}

\subsection{函数模型}

\subsection{误差模型}

\subsection{随机模型}

\section{导航卫星运动模型}

\subsection{动力学方程和状态转移}

\subsection{摄动力模型}
