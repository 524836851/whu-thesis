% Chapter 3

\chapter{GNSS实时滤波精密轨道确定的研究与实现}

\section{引言}

\section{基于平方根信息滤波的参数估计原理}

\subsection{量测更新算法}

\subsection{时间更新算法}

\subsection{整体流程}

\section{GNSS数据的实时质量控制}

\subsection{实时周跳探测算法}

\subsection{实时质量控制算法}

\subsection{实验结果和分析}

实验设计:
主要目的:选用经过多次迭代后的log\underline{\space}tb文件作为参考值,认为其中没有发生周跳。以此对比实时质量算法所起的作用。

分别使用该log\underline{\space}tb文件,以及实时周跳探测和质量控制的算法分别进行实时滤波轨道确定计算

实验方案: 110 个测站 10天的3d解(天数可以商榷) GPS单系统
方案1:使用log\underline{\space}tb进行计算(关掉了实时周跳探测以及实时质量控制,纯解算)

方案2:开启litetb(感觉需要使用300s的阈值,否则这样信息感觉比事后的tb解算多),其次是开启质量控制算法(即重置模糊度,这里需要测试不同max\_res\_norm阈值下的结果吗 这个感觉有点区别啊)

\section{实时双差模糊度固定方法}

\subsection{双差模糊度固定算法基本原理}

\subsection{实验结果和分析}

实验设计:

主要目的:
相较于浮点解,对比不同模糊度固定算法在实时轨道中所起的作用

实验方案:110个测站  10天的3d解,基于编辑残差后的log\_tb进行解算结果? GCE三系统的结果
同时对比不同的实验方案(紧约束的和松约束)

这里就直接说是基于残差log\_tb的结果算了,不过实际算的时候应该使用初始tb加上实时质量控制的算法

\section{实时滤波精密轨道处理软件结构}

画图画图再画图
