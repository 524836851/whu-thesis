% Chapter 3

\chapter{GNSS实时滤波精密轨道确定的研究与实现}

\section{引言}

不论是提供事后或是实时的导航卫星轨道的位置服务,目前主流的GNSS精密轨道处理的方式仍是基于事后批处理的解算模型,其算法模型和处理流程随着多年来研究已经逐步趋于完善。相对的,基于实时观测数据,采用实时滤波解算方式进行轨道确定的处理模式在近几年仍在不断的研究和探索中。在事后批处理过程中,由于包含了所有观测数据信息,因此可以对观测数据进行统一的预处理,同时在解算过程中进行反复迭代计算和整体质量控制,以达到最优的数据处理效果。而实时滤波处理流程中往往需要通过处理实时观测数据流,实时给出当前已有观测数据信息下的最优轨道位置信息。因此实时滤波精密轨道确定在不论是在观测数据的预处理、参数估计方法、质量控制方法以及整体算法流程都与事后解算模式大不相同。如果直接应用事后解算模式的经验与方法,将难以取得理想的解算效果。

本章从实时滤波精密轨道确定的算法流程出发,分别对处理过程中的关键环节:参数估计方法、实时质量控制方法和实时模糊度固定方法,进行了相应的推导和实现,并通过实验对比分析验证了算法的正确性和有效性。最后,本文在目前已有的数据处理软件平台上GREAT(GNSS+ Research,
Application and Teaching)的基础上,开发实现了具有GNSS实时滤波精密轨道确定的功能,并给出了该功能的整体结构组成及其算法处理流程。

\section{基于平方根信息滤波的参数估计原理}

在第二章的参数估计部分,我们介绍了常用的最小二乘算法和卡尔曼滤波估计算法。其中卡尔曼滤波估计算法更适合实时数据的处理,相对于最小二乘批处理需要存储所有的观测值信息,滤波估计算法无需存储历史时刻的观测数据信息,而是以待估参数的协方差矩阵信息进行存储。同时滤波算法在处理具有先验运动模型的最优估计问题也更为直观。但由于计算机中计算过程截断误差的存在,导致存在滤波因数值误差而发散的情况存在,因此引入了平方根滤波相关的算法(Dyer,1969),其核心原理通过采用原有滤波算法一半字节长度进行相关数据信息的存储,大大减少了数值计算误差,从而抑制了滤波发散的情况,具有更高的数值稳定性。这里,我们选用了实时滤波轨道处理中常用的平方根信息滤波(Square Root Information Filter,SRIF)作为参数估计的方法。由于广义最小二乘算法在测绘领域更为常用,因此本文从广义最小二乘算法角度出发,推导和梳理了SRIF算法过程。

\subsection{量测更新算法}

对于一个包含观测误差在内的观测信息

\subsection{时间更新算法}

\subsection{实时滤波轨道处理应用}

\section{GNSS数据的实时质量控制}

画图

\subsection{实时周跳探测算法}

\subsection{实时质量控制算法}

\subsection{实验结果和分析}

实验设计:
主要目的:选用经过多次迭代后的log\underline{\space}tb文件作为参考值,认为其中没有发生周跳。以此对比实时质量算法所起的作用。

分别使用该log\underline{\space}tb文件,以及实时周跳探测和质量控制的算法分别进行实时滤波轨道确定计算

实验方案: 110 个测站 10天的3d解(天数可以商榷) GPS单系统
方案1:使用log\underline{\space}tb进行计算(关掉了实时周跳探测以及实时质量控制,纯解算)

方案2:开启litetb(感觉需要使用300s的阈值,否则这样信息感觉比事后的tb解算多),其次是开启质量控制算法(即重置模糊度,这里需要测试不同max\_res\_norm阈值下的结果吗 这个感觉有点区别啊)

\section{实时双差模糊度固定方法}

\subsection{双差模糊度固定算法基本原理}

\subsection{实验结果和分析}

实验设计:

主要目的:
相较于浮点解,对比不同模糊度固定算法在实时轨道中所起的作用

实验方案:110个测站  10天的3d解,基于编辑残差后的log\_tb进行解算结果? GCE三系统的结果
同时对比不同的实验方案(紧约束的和松约束)

这里就直接说是基于残差log\_tb的结果算了,不过实际算的时候应该使用初始tb加上实时质量控制的算法

\section{实时滤波精密轨道处理软件结构}

画图画图再画图
